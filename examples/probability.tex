\documentclass[11pt, a4paper]{article}
\usepackage[english]{babel}

\usepackage[probability]{../atlas-preamble}

\title{Probability and \texttt{atlas-preamble}}
\author{Responsible for \texttt{probability} module: Tassilo Schwarz\footnote{Email: \texttt{lastname@maths.ox.ac.uk}}}
\date{\today}


\begin{document}

\maketitle


\section{Introduction}
To use probability notation and packages in your document, use \texttt{atlas-preamble}, use the \texttt{probability} option:
\begin{verbatim}
	\usepackage[probability]{atlas-preamble}
\end{verbatim}


\noindent
You then have the following benefits.




\section{Basic probability notation}
The most commonly used one is $\Pr{X+Y}$ and \iid. The \texttt{$\textbackslash$iid} command solves for you every problem with spacing after \texttt{$\textbackslash$iid}: You can either end the sentence afterwards or continue writing; the package will work the spacing out for you:
\begin{itemize}
	\item Let $X$ and $Y$ be \iid. This is a new sentence. 
	\item Let $X$ be \iid and $Y$ be \iid.
\end{itemize}

\noindent
But use \texttt{$\textbackslash$iid} only in text mode. 

\noindent Other elementary probability notation includes:

\begin{align*}
	&\text{Expectation}
	&&\text{Variance}
	&&\text{Covariance} 
	&& r\text{-th Factorial Moment} & \\
	&\texttt{\textbackslash E\{a\}\{a+b\}}
	&&\texttt{\textbackslash Var\{a\}\{a+b\}}
	&&\texttt{\textbackslash Cov\{a\}\{A+u v\^{}T\}} 
	&& \texttt{\textbackslash Efact\{r\}\{X\}} & \\
	&\E{a}{a+b} 
	&& \Var{a}{a+b}
	&& \Cov{a}{A+u v^T} 
	&& \Efact{r}{X} &
\end{align*}
If only one argument is given, the subscript will be left out.




\section{Probability Distributions}

The following probability distributions are implemented.

\begin{align*}
	&\texttt{\textbackslash Bernoulli\{p\}}
	&&\texttt{\textbackslash Bin\{n\}\{p\}}
	&&\texttt{\textbackslash Po\{c\}} &  \\
	&\Bernoulli{p} 
	&& \Bin{n}{p}
	&& \Po{c} &
\end{align*}
\end{document}
