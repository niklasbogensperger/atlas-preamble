\documentclass[11pt, a4paper]{article}

\usepackage[english]{babel}

\usepackage[probability]{../atlas-preamble}


\title{Probability and \texttt{atlas-preamble}}
\author{Responsible for \texttt{probability} preset: Tassilo Schwarz\footnote{Email: \texttt{lastname@maths.ox.ac.uk}}}
\date{\today}



\begin{document}

\maketitle


\section{Introduction}
To use probability theory macros in your document with \texttt{atlas-preamble}, use the \texttt{probability} preset option:
\begin{verbatim}
	\usepackage[probability]{atlas-preamble}
\end{verbatim}


\section{Basic probability notation}
The most commonly used ones are $\Pr{X+Y}$ and \iid. The default \texttt{$\textbackslash$Pr} command ($\oldPr$) is accessible via the alias \texttt{$\textbackslash$oldPr}. The new \texttt{$\textbackslash$Pr} command can also be used with an argument to put an event in square brackets. The \texttt{$\textbackslash$iid} command solves spacing and punctuation problems after \texttt{$\textbackslash$iid} automatically: You can either end the sentence afterwards or continue writing; the package will work everything out for you:
\begin{itemize}
	\item Let $X$ and $Y$ be \iid. This is a new sentence. 
	\item Let $X$ be \iid and $Y$ be \iid.
\end{itemize}

\noindent
Caution: use \texttt{$\textbackslash$iid} only in text mode. 

\noindent Other elementary probability notation includes:
\begin{align*}
	&  \textbf{Expectation}
	&& \textbf{Variance}
	&& \textbf{Covariance} 
	&& r\textbf{-th Factorial Moment}
	&  \\
	&  \texttt{\textbackslash E\{a\}\{a+b\}}
	&& \texttt{\textbackslash Var\{a\}\{a+b\}}
	&& \texttt{\textbackslash Cov\{a\}\{A+u v\^{}T\}} 
	&& \texttt{\textbackslash Efact\{r\}\{X\}}
	&  \\
	&  \E{a}{a+b} 
	&& \Var{a}{a+b}
	&& \Cov{a}{A+u v^T} 
	&& \Efact{r}{X}
	&
	\intertext{If only one argument is given, the subscript will be left out:}
	&  \texttt{\textbackslash E\{a+b\}}
	&& \texttt{\textbackslash Var\{a+b\}}
	&& \texttt{\textbackslash Cov\{A+u v\^{}T\}} 
	&& \texttt{(not available)}
	&  \\
	&  \E{a+b} 
	&& \Var{a+b}
	&& \Cov{A+u v^T} 
	&& \texttt{(not available)}
\end{align*}

\noindent
Parenthesis and brackets scaling is automatic according to the content.


\section{Probability Distributions}

The following macros for probability distributions are available:
\begin{align*}
	&  \texttt{\textbackslash Bernoulli\{p\}}
	&& \texttt{\textbackslash Bin\{n\}\{p\}}
	&& \texttt{\textbackslash Po\{c\}}
	&  \\
	&  \Bernoulli{p} 
	&& \Bin{n}{p}
	&& \Po{c}
\end{align*}

\end{document}
